\documentclass[11pt, oneside]{article}   	% use "amsart" instead of "article" for AMSLaTeX format
\usepackage{geometry}                		% See geometry.pdf to learn the layout options. There are lots.
\geometry{letterpaper}                   		% ... or a4paper or a5paper or ... 
%\geometry{landscape}                		% Activate for rotated page geometry
%\usepackage[parfill]{parskip}    		% Activate to begin paragraphs with an empty line rather than an indent
\usepackage{graphicx}				% Use pdf, png, jpg, or eps§ with pdflatex; use eps in DVI mode
								% TeX will automatically convert eps --> pdf in pdflatex		
\usepackage{amssymb}
\usepackage{hyperref}
\hypersetup{
	colorlinks,
	urlcolor=blue
}
%SetFonts

%SetFonts


\begin{document}

\section*{Comparing contributions of individual changes to their combined effects in DNA sequence}
\author{Griffin Chure, Soichi Hirokawa}

One of the most fascinating features of the immune system in humans is not
only its ability to produce a vast array of antibody types to identify any
invasive bacterium or virus-infected cell, but also the elegance of making
an antibody by cutting and joining different regions of DNA. The
initial cutting part of this process involves a protein called RAG
binding onto two regions of the DNA each of which are adjacent
to antibody-encoding sequences selected for joining. RAG attaches to these
two regions because they exhibit certain sequence patterns that are amenable
for binding. RAG cuts the DNA between these sites and the antibody-encoding
portions before other proteins complete the DNA joining phase for cells to
make antibodies.

While RAG binds and cuts specific regions of the DNA because of the
recognizable sequence patterns, these sites can still vary in sequence within
the genome and may impact its neighboring antibody-encoding region's chances of
being selected to produce the required antibody. In a study that we recently
published in Nucleic Acids Research,
we examined the extent to which RAG will bind and cut the DNA if we modify a
binding site sequence at multiple positions and compared the individual contributions of
each nucleotide change against their collective effect. We illustrate some of
our findings in this visual, which is modified from a page in the
\href{https://www.rpgroup.caltech.edu/vdj_recombination}{Supplementary website for our publication}. In this interactive
visual, we provide three examples comparing effects of several single
nucleotide changes from a common starting sequence and that of combining these
replacements into a single sequence. Through the dropdown menu, one can
select any of these three binding site sequences to reveal the effects of the
sequence and the individual effects of its constitutive changes. The upper
left plot shows the frequency that RAG creates a DNA loop for the combination
of changes to the far right and the individual changes to the left, with
the starting sequence labeling the x-axis. The upper right plot shows full posterior distributions of the probability that
RAG cuts the DNA with the altered sequence. In the bottom row, we present
three cumulative distribution functions to show (from left to right) how much
time it takes before DNA unloops without cutting, time before a loop is cut, or a compilation of
the two possible fates. To more easily compare one particular single
nucleotide change against the combined changes, hovering the mouse over a
colored nucleotide in the sequence below the dropdown menu will send the rest
of the data into the background and present only the individual change with
the superposition of changes.

Some nucleotides can have a dominating influence on how well RAG binds or
cuts the DNA. For the sequence called V8-18, the change from T to A at
position six prevents RAG from binding these sites. Antibody-encoding DNA
segments neighboring binding sites with this T-to-A change are rarely
selected which in some cases creates significant obstacles to making the antibody
necessary for fighting off some infections.

\end{document}  