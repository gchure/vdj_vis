\documentclass[11pt, oneside]{article}   	% use "amsart" instead of "article" for AMSLaTeX format
\usepackage{geometry}                		% See geometry.pdf to learn the layout options. There are lots.
\geometry{letterpaper}                   		% ... or a4paper or a5paper or ... 
%\geometry{landscape}                		% Activate for rotated page geometry
%\usepackage[parfill]{parskip}    		% Activate to begin paragraphs with an empty line rather than an indent
\usepackage{graphicx}				% Use pdf, png, jpg, or eps§ with pdflatex; use eps in DVI mode
								% TeX will automatically convert eps --> pdf in pdflatex		
\usepackage{amssymb}

%SetFonts

%SetFonts


\begin{document}

\section*{Synthetic-endogenous RSS comparison tool}
\author{Griffin Chure, Soichi Hirokawa}

Arguably one of the most fascinating feats of the immune system in humans, and really in any jawed vertebrate, is not only
its ability to produce a seemingly endless array of antibody types to identify any invasive bacterium or virus-infected cell,
but also the elegance of making antibody assembly a matter of cutting and pasting different regions of DNA together. This
assembly called V(D)J recombination keeps the genome from having an unwieldy number of antibody genes and is one of
the clever ways in which immune cells efficiently call upon the desired combination of antibody components to correctly
identify the foreign substance. The initial cutting part of V(D)J recombination involves the recombination-activating gene (RAG)
enzyme bringing together two short, recognizable DNA sequences called recombination signal sequences (RSSs) adjacent 
to two of these gene segments before cleaving the DNA between them. A broad area of interest in the field is what factors
are responsible for cells producing some antibody types more often than others. Using an experimental method that allows for
detection of the formation, lifetime, and cutting of this RAG-RSS complex, we examined how RSS sequence, which despite
some key characteristics can differ by a few base pairs from one another, contributes to the chances of the neighboring gene
segment being selected for recombination.

\end{document}  