\documentclass[11pt, oneside]{article}   	% use "amsart" instead of "article" for AMSLaTeX format
\usepackage{geometry}                		% See geometry.pdf to learn the layout options. There are lots.
\geometry{letterpaper}                   		% ... or a4paper or a5paper or ... 
%\geometry{landscape}                		% Activate for rotated page geometry
%\usepackage[parfill]{parskip}    		% Activate to begin paragraphs with an empty line rather than an indent
\usepackage{graphicx}				% Use pdf, png, jpg, or eps§ with pdflatex; use eps in DVI mode
								% TeX will automatically convert eps --> pdf in pdflatex		
\usepackage{amssymb}
\usepackage{hyperref}
\hypersetup{
	colorlinks,
	urlcolor=blue
}
%SetFonts

%SetFonts


\begin{document}

\section*{Synthetic-endogenous RSS comparison tool}
\author{Griffin Chure, Soichi Hirokawa}

Arguably one of the most fascinating feats of the immune system in humans, and really in any jawed vertebrate, is not only
its ability to produce a seemingly endless array of antibody types to identify any invasive bacterium or virus-infected cell,
but also the elegance of making antibody assembly a matter of cutting and pasting different regions of DNA together. This
assembly called V(D)J recombination involves taking DNA sequences encoding parts of these antibodies and stitching them
together and has proven to be one of the clever ways in which organisms keep the genome from taking on an unwieldy 
number of antibody genes while producing the necessary antibody on demand. The initial cutting part of V(D)J recombination 
involves the recombination-activating gene (RAG) protein, as anthropomorphized into hands in the figure, creating a loop 
in the DNA by grabbing onto two regions of the DNA, shown with purple and blue backbones, adjacent to these 
antibody-encoding segments so that they may be later joined. These two regions exhibit a series of sequence patterns 
that is amenable to RAG binding. RAG will eventually cut the DNA between these sites and the antibody-encoding portions
for other proteins to join the antibody gene segments to yield an antibody combination.

While RAG manages to attach to and cut certain regions of the DNA because of the recognizable sequence patterns, these
identifiable patterns are interspersed among DNA sequence positions that have less obvious patterns. In a study that was 
recently published in \emph{Nucleic Acids Research}, we set out to understand how switching out nucleotides at one or
multiple positions of these binding sites from a common initial sequence alters the extent to which RAG will hold the 
DNA and cut it. We illustrate some of our findings in this interactive figure, which is modified from a page in 
\href{https://www.rpgroup.caltech.edu/vdj_recombination}{the Supplementary website that accompanies our publication}. 
In this interactive figure, we show data from three naturally occurring binding site sequences and compare them to
data from the initial binding site sequence and data collected on a single nucleotide change that accounts for the sequence
difference between the initial sequence and the naturally-occurring one. Through the dropdown menu, one can select
one of these three binding site sequences to reveal its sequence, with positions where the nucleotide is changed from
the initial sequence colored in. The plot immediately below it shows the frequency that RAG creates a DNA loop with

For simplicity, we also make it easier to identify how a single nucleotide change compares to the naturally occurring
sequence of which it contributes to the sequence change. Hovering the mouse over one of the colored nucleotides will send
the rest of the data into the background for ease of visualizing data from the single nucleotide change and the naturally
occurring sequence.

The data we present here shows that some nucleotides can have a dominating influence on how well RAG can hold or
cut the DNA.

\end{document}  